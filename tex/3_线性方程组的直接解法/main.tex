\documentclass[UTF8, a4paper, zihao=-4, bibliography=totoc]{ctexart}
\usepackage{color}
\usepackage[dvipsnames]{xcolor}
\usepackage{mathtools}
\usepackage[colorlinks=true, pdfstartview=FitH, linkcolor=black, anchorcolor=violet, citecolor=magenta]{hyperref}
\usepackage{graphicx}
\usepackage{enumitem}
\usepackage{titlesec}
\usepackage{titletoc}
\usepackage{bm}
\usepackage{amsmath}
\usepackage{amsfonts}
\usepackage{mathrsfs}
\usepackage{multirow}
\usepackage{amsthm}
\usepackage{tabularx, booktabs}
\usepackage{longtable}
\usepackage{supertabular}
\usepackage{array, makecell}
\usepackage{graphicx}
\usepackage{epstopdf}
\usepackage{float}
\usepackage{subfigure}
\usepackage{geometry}
\usepackage{ctex}
\usepackage{listings}
\usepackage[nottoc,numbib]{tocbibind}
\usepackage[backend=biber, style=gb7714-2015]{biblatex}
\addbibresource[location=local]{../reference.bib}
\usepackage{pythonhighlight}
\usepackage{caption}
\usepackage{booktabs}
\usepackage{algorithm}
\usepackage{algorithmic}
\usepackage[toc]{appendix}
\usepackage{xhfill}
%%%%%%%%%%%%%%%%%%%%%%%%%%%%%%%%%%%%%%%%%%%%%%%%%%%%%%%%%%%%%%%%%%%%%%%%%%%%%%%%
%%%%%%%%%%%%%%%%%%%%%%%%%%%%%%%%% ctexset %%%%%%%%%%%%%%%%%%%%%%%%%%%%%%%%%%%%%%
%%%%%%%%%%%%%%%%%%%%%%%%%%%%%%%%%%%%%%%%%%%%%%%%%%%%%%%%%%%%%%%%%%%%%%%%%%%%%%%%
\ctexset{
    bibname     =   {参考文献},
    section = {
        number  =   \arabic{section},
        format  +=  \zihao{-4}\raggedright,
        name    =   {,.},
    },
    subsection = {
        format  =  \bf{\zihao{-4}}\raggedright,
    },
    subsubsection = {
        number  =   \arabic{subsubsection},
        format  +=  {\zihao{-4}}\raggedright,
    }
}

% 用来设置附录中代码的样式

\definecolor{mygreen}{RGB}{28,172,0} % color values Red, Green, Blue
\definecolor{mylilas}{RGB}{170,55,241}

\lstset{
    basicstyle          =   \sffamily,
    keywordstyle        =   \bfseries,
    commentstyle        =   \rmfamily\itshape,
    stringstyle         =   \ttfamily,
    flexiblecolumns,
    numbers             =   left,
    showspaces          =   false,
    numberstyle         =   \zihao{-5}\ttfamily,
    showstringspaces    =   false,
    captionpos          =   t,
    frame               =   lrtb,
}

\lstdefinestyle{Python}{
    language        =   Python,
    basicstyle      =   \zihao{-5}\ttfamily,
    numberstyle     =   \zihao{-5}\ttfamily,
    keywordstyle    =   \color{blue},
    keywordstyle    =   [2] \color{teal}, % just to check that it works
    stringstyle     =   \color{magenta},
    commentstyle    =   \color{mygreen}\ttfamily,
    breaklines      =   true,
    columns         =   fixed,
    basewidth       =   0.5em,
}

\lstdefinestyle{MATLAB}{
    language        =   Matlab,
    basicstyle      =   \zihao{-5}\ttfamily,
    numberstyle     =   \zihao{-5}\ttfamily,
    keywordstyle    =   \color{blue},
    keywordstyle    =   [2] \color{teal}, % just to check that it works
    stringstyle     =   \color{mylilas},
    commentstyle    =   \color{mygreen}\ttfamily,
    breaklines      =   true,
    columns         =   fixed,
    basewidth       =   0.5em,
}

\theoremstyle{definition}

\newtheorem{definition}{定义}[section]

\newtheorem{theorem}{定理}[section]

\newtheorem{corollary}{推论}[theorem]

\newtheorem{lemma}[theorem]{引理}

\newenvironment{myquote}
    {\begin{quote}\kaishu\zihao{5}}
    {\end{quote}}

\setcounter{MaxMatrixCols}{20}

\renewcommand{\algorithmicrequire}{ \textbf{Input:}} %Use Input in the format of Algorithm 
\renewcommand{\algorithmicensure}{ \textbf{Output:}} %UseOutput in the format of Algorithm
% 定义页面边距

\newgeometry{
    left=2.5cm,
    right=2.0cm,
    top=2.5cm,
    bottom=2.5cm
}

%%%%%%%%%%%%%%%%%%%%%%%%%%%%%%%%%%%%%%%%%%%%%%%%%%%%%%%%%%%%%%%%%%%%%%%%%%%%%%%%
%%%%%%%%%%%%%%%%%%%%%%%%%%%%%%%%%%% document %%%%%%%%%%%%%%%%%%%%%%%%%%%%%%%%%%%
%%%%%%%%%%%%%%%%%%%%%%%%%%%%%%%%%%%%%%%%%%%%%%%%%%%%%%%%%%%%%%%%%%%%%%%%%%%%%%%%

\begin{document}

\newcommand{\ThisProjectTitle}{线性方程组的直接解法}
\newcommand{\ThisDate}{2017-10-16}
\newcommand{\ThisNo}{No.1}

\newcommand{\CourseName}{{\bf 课程名称:}}
\newcommand{\Grade}{{\bf 年级:}}
\newcommand{\Score}{{\bf 上机实践成绩:}}
\newcommand{\Director}{{\bf 指导教师:}}
\newcommand{\StudentName}{{\bf 学生姓名:}}
\newcommand{\ProjectTitle}{{\bf 上机实践名称:}}
\newcommand{\StudentID}{{\bf 学号:}}
\newcommand{\Date}{{\bf 上机实践日期:}}
\newcommand{\No}{{\bf 上机实践编号:}}
\newcommand{\GroupNum}{{\bf 组号:}}
\newcommand{\LastEditTine}{{\bf 最后修改时间:}}

\newcommand{\ThisCourseName}{数值计算实验}
\newcommand{\MyGrade}{2015级}
\newcommand{\MyScore}{100}
\newcommand{\MyDirector}{朱娟萍}
\newcommand{\MyName}{刘鹏}
\newcommand{\ThisProjectTitle}{非线性方程求根}
\newcommand{\MyID}{20151910042}
\newcommand{\ThisDate}{2017-10-16}
\newcommand{\ThisNo}{No.1}
\newcommand{\MyGroupNum}{1}
\newcommand{\MyLastEditTine}{\today}

\newcommand{\HRule}{\rule{\linewidth}{0.3mm}}
\newcommand{\xfill}[2][1ex]{{%
  \dimen0=#2\advance\dimen0 by #1
  \leaders\hrule height \dimen0 depth -#1\hfill%
}}
\newcommand{\xfilll}[2][1ex]{%
  \dimen0=#2\advance\dimen0 by #1%
  \leaders\hrule height \dimen0 depth -#1\hfill%
}

\begin{center}
    {\zihao{3} \bf 云南大学数学与统计学院}\\
    {\zihao{3} \bf 上机实践报告}
\end{center}

\begin{table}[h]
    \centering
    \resizebox{\textwidth}{!}{%
        \begin{tabular}{|l|l|l|}
        \hline
            \CourseName \ThisCourseName     & \Grade \MyGrade       & \Score                        \\ \hline
            \Director \MyDirector           & \StudentName \MyName  &                               \\ \hline
            \ProjectTitle \ThisProjectTitle & \StudentID \MyID      & \Date \ThisDate               \\ \hline
            \No \ThisNo                     & \GroupNum             & \LastEditTine \MyLastEditTine \\ \hline
        \end{tabular}
    }
    \xfill{30pt}
\end{table}

\section{实验目的}

1. 通过对所学的线性方程组直接求解的理论方法进行编程,提升程序编写水平;

2. 通过对理论方法的编程实验,进一步掌握理论方法的每一个细节;

3. 通过数值法求解,发现数值方法与符号方法的区别,并形成专业思维。

\section{实验内容}

1. 编程实现高斯-若尔当列主元消元法;

2. 编程实现高斯-若尔当全主元消元法;

3. 任选一种方案,Doolittle分解或者Crout分解,编程实现矩阵的LU分解;

4. 编程实现三对角线矩阵的稀疏方式存储,然后对其进行LU分解。

\section{实验平台}

macOS

Python 3.7.3;

MATLAB R2017b win64;

\section{实验记录与实验结果分析}

\subsection{第1题}

1题
编程实现:用高斯-若尔当列主元消元法求下列方程的解[1]:

\begin{equation}
    \left\{\begin{aligned}
        x_{1}+2 x_{2}+x_{3}     &=  2   \\
        -2 x_{1}-2 x_{2}-x_{3}  &=  -3  \\
        2 x_{1}-3 x_{2}-2 x_{3} &=  -1 
    \end{aligned}\right.
\end{equation}

\subsubsection{程序代码}

\lstinputlisting[
    style       =   Python,
    %caption     =   {\bf ff.py},
    label       =   {get_root.py}
]{../../src/3_线性方程组的直接解法/ColumnPivotMethod.py}

\subsubsection{运行结果}

\subsubsection{结果分析}

由于二分法与埃特金方法的函数并不是一样的,前者是原函数,后者是迭代函数,所以很难写一个通用算法解决这个迭代函数的生成问题。所以这个class意义不是很大,不过可以通过对属性进行赋值,重复进行计算,也算有一定的灵活性。

\section{实验体会}

通过编程,复习了简单迭代法及其改进。明白了二分法与埃特金法的斯坦弗森过程之间的区别。

\section{参考文献}

[1] 金一庆, 陈越, 王冬梅. 数值方法[M]. 北京: 机械工业出版社; 2000.2.


\section{实验目的}

1. 通过对所学的线性方程组直接求解的理论方法进行编程,提升程序编写水平;

2. 通过对理论方法的编程实验,进一步掌握理论方法的每一个细节;

3. 通过数值法求解,发现数值方法与符号方法的区别,并形成专业思维。

\section{实验内容}

1. 编程实现高斯-若尔当列主元消元法;

2. 编程实现高斯-若尔当全主元消元法;

3. 任选一种方案,Doolittle分解或者Crout分解,编程实现矩阵的LU分解;

4. 编程实现三对角线矩阵的稀疏方式存储,然后对其进行LU分解。

\section{实验平台}

macOS

Python 3.7.3;

MATLAB R2017b win64;

\section{实验记录与实验结果分析}

\subsection{第1题}
\begin{quote}
    {\kaishu
        编程实现:用高斯-若尔当列主元消元法求下列方程的解[1]:
    }
    \begin{equation}
        \left\{\begin{aligned}
            x_{1}+2 x_{2}+x_{3}     &=  2   \\
            -2 x_{1}-2 x_{2}-x_{3}  &=  -3  \\
            2 x_{1}-3 x_{2}-2 x_{3} &=  -1 
        \end{aligned}\right.
    \end{equation}
\end{quote}

\subsubsection{程序代码}

\lstinputlisting[
    style       =   Python,
    %caption     =   {\bf ff.py},
    label       =   {get_root.py}
]{../../src/3_线性方程组的直接解法/ColumnPivotMethod.py}

\subsubsection{运行结果}

\subsubsection{结果分析}

本Python代码采用面向对象[2]的方式写成,首先是定义了一个Matrix类,用来存储矩阵,该类中的所有元素都是public的,这样设置是为了以后的编程调用方便。

Matrix的实例A是一个三行四列的矩阵,其中最后一列是方程组中等号右边的元素,所以这个增广矩阵的计算应该遵循线性代数的法则,即在寻找列主元的时候,不能越界去增广列寻找;同样,一个已经找出过列主元的行,不应该再次存在列主元,所以要在以后的查找中剔除这一行。

最后,输出的结果应该保持美观性,所以按照系数矩阵经过变换之后为单位矩阵的样式进行了重新排序。

值得指出的是,Matrix的matrixTransform成员函数是用多态的方式写的,可以满足行数乘、行交换、行倍加等所有的矩阵初等变换方式。

另外,由于本线性方程组一定存在唯一解,所以在排序与寻找主元的过程中,并没有加以判断,所以这并不是一个通用的程序。

\subsection{第2题}
\begin{quote}
    {\kaishu
        编程实现:请用高斯-若尔当全主元消元法求下列矩阵的逆矩阵[1]:
    }
    \begin{equation}
        \mathbf{B} = \left[\begin{array}{rrrr}
            2   & 1  & -3   & -1 \\
            3   & 1  & 0    & 7  \\
            -1  & 2  & 4    & -2 \\
            1   & 0  & -1   & 5
        \end{array}\right]
    \end{equation}
\end{quote}

\subsubsection{程序代码}
\lstinputlisting[
    style       =   Python,
    %caption     =   {\bf ff.py},
    label       =   {get_root.py}
]{../../src/3_线性方程组的直接解法/CompletePivotMethod.py}

\subsubsection{代码分析}
此段代码是在Code Box 1的基础上略微修改而得的,由于全主元的位置没有固定性,所以要把每一次找到的全主元的行列坐标都存储下来。其余部分与Code Box 1相同。

\subsection{第3题}
\begin{quote}
    {\kaishu
        编程实现:利用LU分解法(Doolittle),求$[x_1,\ x_2,\ x_3,\ x_4]^\mathrm{T}$的数值解。
    }
    \begin{equation}
        \left[ \begin{array}{rrrr}
            {12} & {-3} & {3} & {4} \\ {-18} & {3} & {-1} & {-1} \\ {1} & {1} & {1} & {1} \\ {3} & {1} & {-1} & {1}
        \end{array}\right] \cdot \left[ \begin{array}{c}{x_{1}} \\ {x_{2}} \\ {x_{3}} \\ {x_{4}}\end{array}\right]=\left[ \begin{array}{r}{15} \\ {-15} \\ {6} \\ {2}\end{array}\right]
    \end{equation}
\end{quote}

\subsubsection{程序代码}
\lstinputlisting[
    style       =   Python,
    %caption     =   {\bf ff.py},
    label       =   {get_root.py}
]{../../src/3_线性方程组的直接解法/LU_Decomposition.py}

\subsubsection{代码分析}

通过生成的LU矩阵,可以轻易得到, 的数值解为 ,这也是解析解。

代码仍旧是基于面向对象技术,程序的算法是基于紧凑法求解LU矩阵的形式化做法。

如此编程,相对而言比较节约内存。

\subsection{第4题}
\begin{quote}
    {\kaishu
        编程实现:用追赶法解下列严格对角优势的三对角线方程组,要求用稀疏格式存储矩阵,主内存占用为$\mathrm{O}(n)$,其中$n$为矩阵的行数。
    }
    \begin{equation}
        \left[ \begin{array}{rrrrr}
            {4} & {-1} & {0} & {0} & {0} \\ {-1} & {4} & {-1} & {0} & {0} \\ {0} & {-1} & {4} & {-1} & {0} \\ {0} & {0} & {-1} & {4} & {-1} \\ {0} & {0} & {0} & {-1} & {4}\end{array}\right] \cdot \left[ \begin{array}{c}{x_{1}} \\ {x_{2}} \\ {x_{3}} \\ {x_{4}} \\ {x_{5}}\end{array}\right]=\left[ \begin{array}{c}{100} \\ {200} \\ {200} \\ {200} \\ {200} \\ {100}
        \end{array}\right]
    \end{equation}
\end{quote}

\subsubsection{程序代码}
\lstinputlisting[
    style       =   Python,
    %caption     =   {\bf ff.py},
    label       =   {get_root.py}
]{../../src/3_线性方程组的直接解法/Seize_Method.py}

\section{实验体会}

在此次实验中,集中利用了MATLAB与Python 3进行程序设计。

主程序是采用Python 3进行编写,验算求解是利用MATLAB进行。当写的程序足够多之后,就会发现,真正重要的东西是数学思维,编程完全决定于数学思维的深度。如果一味追求编程的快感,很容易在这虚无的成就感中迷失自己。所以编程不如不编,能透彻理解数学思维已经很不容易了。

但是从另一个角度看,编程对数学思维又有很强的检验作用,尽管在数学思维之外,高技巧度编程又是另外广阔的一片天地。通过适度的编程,选择一类题目的典型例子进行编程,就可以在一个程序的设计、实现时间里,考验自己的一批数学思维的掌握牢固程度。

MATLAB的强大之处不在于它编程方便,而是两方面:友好的交互式控制台,众多高质量的程序包。正是这两点,使得MATLAB在工业中得到广泛的应用。但是在数值计算的学习上,利用Python或许是更好的选择。Python的数值计算依赖于数组,所以对于循环的考验非常高。Python不像MATLAB的基本运算都是基于矩阵那样地方便,而正是这不方便这一点,使得Python这个看起来更加拙劣的工具更适合新手。

除了工具,更重要的一点是编程。这四个程序的编写基本上没有遇到困难。如果说在理论方面已经理解得足够透彻,那么在编程上不应该存在任何思路困难。数值计算以及数学建模中的程序,很大程度上都是数学语言的直接翻译。这一点与DSA设计有相当大的差别。现在出现的一个问题是,编程趋向IDD模式,即IDE Driven,没有了编译器就不会单步调试,进而很难发现哪里出了差错。这不能说是思路理解得不到位,因为如果单纯是这个原因,那么在纸上演算就会遇到问题。但是很显然不是这样。以后的一个突破,可能就在这里——直接用文本编辑器进行代码书写,然后用IDE进行检验。\cite{RN686}

最后需要说明,因为有很多东西没有或者不能调用,所以这四个程序基本上是为题目量身定做。虽然面向对象设计里的几个成员函数具有通用性,但是总得来看,一些全局函数还是缺乏相应的判断。当然,如果不能运行,就从侧面说明输入的矩阵不合格。所以,这四个程序的通用性有待进一步挖掘,数据与程序能彻底分开,是判断程序通用性强弱的主要依据之一。

\section{参考文献}

\printbibliography[heading=none]

\end{document}
