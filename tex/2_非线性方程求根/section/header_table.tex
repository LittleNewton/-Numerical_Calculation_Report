\newcommand{\CourseName}{{\bf 课程名称:}}
\newcommand{\Grade}{{\bf 年级:}}
\newcommand{\Score}{{\bf 上机实践成绩:}}
\newcommand{\Director}{{\bf 指导教师:}}
\newcommand{\StudentName}{{\bf 学生姓名:}}
\newcommand{\ProjectTitle}{{\bf 上机实践名称:}}
\newcommand{\StudentID}{{\bf 学号:}}
\newcommand{\Date}{{\bf 上机实践日期:}}
\newcommand{\No}{{\bf 上机实践编号:}}
\newcommand{\GroupNum}{{\bf 组号:}}
\newcommand{\LastEditTine}{{\bf 最后修改时间:}}

\newcommand{\ThisCourseName}{数值计算实验}
\newcommand{\MyGrade}{2015级}
\newcommand{\MyScore}{100}
\newcommand{\MyDirector}{朱娟萍}
\newcommand{\MyName}{刘鹏}
\newcommand{\ThisProjectTitle}{非线性方程求根}
\newcommand{\MyID}{20151910042}
\newcommand{\ThisDate}{2017-12-25}
\newcommand{\ThisNo}{No.6}
\newcommand{\MyGroupNum}{1}
\newcommand{\MyLastEditTine}{\today}

\newcommand{\HRule}{\rule{\linewidth}{0.3mm}}
\newcommand{\xfill}[2][1ex]{{%
  \dimen0=#2\advance\dimen0 by #1
  \leaders\hrule height \dimen0 depth -#1\hfill%
}}
\newcommand{\xfilll}[2][1ex]{%
  \dimen0=#2\advance\dimen0 by #1%
  \leaders\hrule height \dimen0 depth -#1\hfill%
}

\begin{center}
    {\zihao{3} \bf 云南大学数学与统计学院}\\
    {\zihao{3} \bf 上机实践报告}
\end{center}

\begin{table}[h]
    \centering
    \resizebox{\textwidth}{!}{%
        \begin{tabular}{|l|l|l|}
        \hline
            \CourseName \ThisCourseName     & \Grade \MyGrade       & \Score                        \\ \hline
            \Director \MyDirector           & \StudentName \MyName  &                               \\ \hline
            \ProjectTitle \ThisProjectTitle & \StudentID \MyID      & \Date \ThisDate               \\ \hline
            \No \ThisNo                     & \GroupNum             & \LastEditTine \MyLastEditTine \\ \hline
        \end{tabular}
    }
    \xfill{30pt}
\end{table}

\section{实验目的}

1. 通过对所学的非线性方程求根法的理论方法进行编程,提升程序编写水平;

2. 通过对理论方法的编程实验,进一步掌握理论方法的每一个细节;

\section{实验内容}

1. 编制求线性方程根的程序;

2. 编程实现用埃特金法求方程的根。

\section{实验平台}

macOS

Python 3.7.3;

MATLAB R2017b win64;

\section{实验记录与实验结果分析}

\subsection{第1题}

用二分法求方程$x^2-x-1=0$的正根,要求精确到小数点后一位。[1]

\subsection{第2题}

请用埃特金方法编程求出$x=\tan⁡x$在$x=4.5$附近的根。

解答:

由于程序比较简单,所以不铺张开写,两题合并写,而且可以对比,同一道题的迭代深度。
\subsubsection{程序代码}

\lstinputlisting[
    style       =   Python,
    %caption     =   {\bf ff.py},
    label       =   {get_root.py}
]{../../src/2_非线性方程求根/get_root.py}

\subsubsection{运行结果}

\subsubsection{结果分析}

由于二分法与埃特金方法的函数并不是一样的,前者是原函数,后者是迭代函数,所以很难写一个通用算法解决这个迭代函数的生成问题。所以这个class意义不是很大,不过可以通过对属性进行赋值,重复进行计算,也算有一定的灵活性。

\section{实验体会}

通过编程,复习了简单迭代法及其改进。明白了二分法与埃特金法的斯坦弗森过程之间的区别。

\section{参考文献}

[1] 金一庆, 陈越, 王冬梅. 数值方法[M]. 北京: 机械工业出版社; 2000.2.
